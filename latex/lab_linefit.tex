% dvips -t letter lab_linefit.dvi -o lab_linefit.ps ; ps2pdf lab_linefit.ps
\documentclass[11pt,titlepage,fleqn]{article}

\input{hw627_header}

\renewcommand{\baselinestretch}{1.0}

\newcommand{\vertgap}{\vspace{1cm}}

% change the figures to ``Figure L3'', etc
\renewcommand{\thefigure}{L\arabic{figure}}
\renewcommand{\thetable}{L\arabic{table}}
\renewcommand{\theequation}{L\arabic{equation}}
\renewcommand{\thesection}{L\arabic{section}}

%--------------------------------------------------------------
\begin{document}
%-------------------------------------------------------------

\begin{spacing}{1.2}
\centering
{\large \bf Lab Exercise: Fitting a line to scattered data} \\
GEOS 626: Applied Seismology, Carl Tape \\
GEOS 627: Inverse Problems and Parameter Estimation, Carl Tape \\
Last compiled: \today
\end{spacing}

%------------------------

\subsection*{Overview}

The objective of this lab is to introduce the concepts of forward and inverse problems by exploring the example of fitting a line to scattered data. We also want to develop some basic capabilities in Matlab, such as vector-matrix operations and plotting.

\begin{enumerate}
\item Write down the forward model $d_i = m_1 + m_2 x_i$ in schematic matrix form $\bd = \bG\bem$.

\vertgap
\vertgap

\item Run the script \verb+lab_linefit.m+ and identify the key parts of the code associated with the forward problem and the inverse problem. What is the Matlab syntax for $\bem^T$, $\bG\bem$, and $\bG^T\bd$?

\vertgap

\item How are the ``target'' (or ``true'') data generated?

\vertgap

\item What are the different ways in Matlab to compute the solution vector $\bem$? \\
(Type \verb+help \+ or \verb+help inv+ for details.)

\vertgap

\item How does the estimated model $\bem_{\rm est}$ compare with the target model $\bem_{\rm tar}$?

\vertgap

How does the standard deviation of the residuals compare with the assumed value of $\sigma$?

\vertgap

\item What does the histogram show? \\
How does it change if you increase the number of points to $10^5$? \\
How does it change if you set $\sigma = 0$?

\vertgap

\item Re-set \verb+n=50+ and $\sigma = 0.3$ and add one outlier to the simulated observations. This can be done by replacing the first observation point with a value that is $1000\sigma$ larger. What happens (and why)?

\vertgap

\pagebreak
\item Now comment out the outlier. Comment out the \verb+break+ statement, then run the last block of code. Make sure you understand what is being plotted and how it is calculated.
%
\begin{itemize}
\item What is the dimension of model space?
\item Fill out the table below.
\item In the code, why is the operation \verb+res.*res+ used instead of \verb+res*res+, \verb+res'*res+, or \verb+res*res'+?
\end{itemize}

\vertgap
\begin{tabular}{c|c|c|c}
\hline
variable & matlab variable  & dimension      & description \\ \hline\hline
         & \verb+d+        &                &              \\ \hline
         & \verb+G+        &                &              \\ \hline
F(\bem)  & \verb+RSS+      &                &              \\ \hline
---      & \verb+RSSm+     &                &              \\ \hline
\bem     & \verb+mtry+     & $2 \times 1$   & \hspace{3cm} \\ \hline
         & \verb+dtry+     &                &              \\ \hline
         & \verb+res+      &                &              \\ \hline
\bgamma(\bem)  & \verb+gammam+ & $2 \times n_g$ & gradient of misfit function \\ \hline
\end{tabular}

\item Referring to the class notes ``Taylor series and least squares'' (\verb+notes_taylor.pdf+) compute the gradient $\bgamma(\bem)$ for the same grid of $\bem$ that was used to plot the misfit function.

Hint: one line of code (plus one line to initialize $\bgamma(\bem)$)

Matlab's \verb+gradient+ function will not help you here, since you need an exact evaluation of the gradient, whereas Matlab will provide a numerical gradient for a grid of pre-computed values.

\vertgap

\item Using the \verb+quiver+ command, superimpose the vector field $-\gamma(\bem)$ on the contour plot of the misfit function.

Set \verb+nx = 10+ so that the vector field is more visible. \\
Hint: This is one line of code. \\
What is the relationship between the contours of $F(\bem)$ and $\bgamma(\bem)$?

\vertgap

How does $\|\bgamma(\bem)\|$ vary with respect to $\bem_{\rm est}$? \\
% it gets larger 
Bonus: Use \verb+surf+ to plot $\|\bgamma(\bem)\|$. What shape is it?
% a cone

\vertgap

\item What is the dependence of the Hessian on the model, $\bH(\bem)$?
% Hessian does not depend on m

\vertgap

What does that imply about $F(\bem)$?
% it is strictly quadratic

%What does that imply about the forward model $\bg(\bem)$?

\end{enumerate}

%-------------------------------------------------------------

% \clearpage\pagebreak

% \small
% \begin{spacing}{1.0}
% \begin{verbatim}
% % lab_linefit.m
% % Carl Tape, GEOS 627, Inverse Problems and Parameter Estimation
% %
% % This program introduces the least squares method for the example of
% % fitting a line (i.e., a model with two parameters) to a set of scattered
% % data.  We show three options for solving the problem, each which gives
% % the same result. No regularization is needed (or used) in this example.

% clear
% close all

% % USER PARAMETERS (CHANGE THESE)
% n = 100;                % number of observations
% sigvec = [0 0.3];       % standard deviations of added errors

% % TARGET model vector (y-intercept, slope)
% mtar = [2.1 -0.5]';
% m = length(mtar);

% % compute design matrix
% x = linspace(-2,2,n)';  % input x-values
% G = [ones(n,1) x];      % n by m design matrix
% N = inv(G'*G)*G';       % m by n data resolution matrix
%                         % (notice the matlab comment associated with inv)
                         
% % display dimensions of these variables
% whos

% nsig = length(sigvec);
% for kk = 1:nsig

%     % generate errors -- these will be different for each run
%     sigma = sigvec(kk);
%     e = sigma * randn(n,1); % normally distributed random numbers
                            
%     % generate target 'data'
%     d = G*mtar + e;
    
%     % optional: add one big anomaly
%     %d(1) = d(1) + 1000*sigma;

%     % SOLVE: compute least squares solution, estimates, and estimated variance.
%     % (We show three options for mest, each with the same result.)
%     mest = G\d              % estimated model
%     %mest = N*d;
%     %mest = flipud(polyfit(x,d,1)')
    
%     dest = G*mest;          % estimated predictions
%     res = d - dest;         % residuals

%     figure; msize = 10;
%     stres = [' std(res) = ' sprintf('%.3f', std(res) )];

%     subplot(2,1,1); hold on;
%     plot(x,d,'.','markersize',msize);
%     plot(x,dest,'r--','linewidth',2);
%     xlabel(' x'); ylabel(' d');
%     title({sprintf('Estimated model : m = (%.2f, %.2f)',mest(1),mest(2)), stres})
%     grid on; axis equal;
%     axis([min(x) max(x) min(G*mtar)-2*sigma max(G*mtar)+2*sigma]);

%     subplot(2,2,3);
%     plot(res,'.','markersize',msize); grid on; ylim([-1 1]);
%     xlabel(' Observation index'); ylabel(' Residual, d - dest'); title(stres);

%     subplot(2,2,4);
%     edges = [-1.05:0.1:1.05]; [Nh,bin] = histc(res,edges);
%     bar(edges,Nh,'histc'); xlim([min(edges) max(edges)]);
%     xlabel(' Residual'); ylabel(' Number'); title([' Ntotal = ' num2str(n)]);

%     %fontsize(11); orient tall, wysiwyg
% end

% break

% %---------------------------
% % generate a plot showing the RSS as a function of model space

% % search range, measured by the distance from the target model
% m1_ran = 1;
% m2_ran = 1;

% % number controlling the number of gridpoints in model space
% nx = 100;
% %nx = 10;   % for gradient plot

% % generate grid for the model space
% m1_vec = linspace(mtar(1)-m1_ran, mtar(1)+m1_ran, nx);
% m2_vec = linspace(mtar(2)-m2_ran, mtar(2)+m2_ran, nx);
% [X,Y] = meshgrid(m1_vec,m2_vec);
% [a,b] = size(X);
% m1 = reshape(X,a*b,1);
% m2 = reshape(Y,a*b,1);

% % compute misfit function (and gradient)
% G = [ones(n,1) x];              % design matrix
% RSS = zeros(n,1);               % initialize misfit function

% for kk=1:a*b
%     mtry = [m1(kk) m2(kk)]';    % a sample from model space
%     dtry = G*mtry;              % predictions from the model
%     res = d - dtry;             % residuals between data and predictions
%     RSS(kk) = sum(res.*res);    % residual sum of squares
    
%     % COMPUTE GRADIENT HERE

% end
% Z = reshape(RSS,a,b);           % reshape for plotting

% % plot the misfit function
% nc = 30;    % number of contours to plot
% figure; hold on;
% contourf(X,Y,Z,nc); shading flat;
% %scatter(m1,m2,6^2,RSS,'filled'); shading flat;
% l1 = plot(mtar(1),mtar(2),'ws','markersize',10,'markerfacecolor','k');
% l2 = plot(mest(1),mest(2),'wo','markersize',10,'markerfacecolor','r');
% legend([l1,l2],'target model','estimated model');
% axis equal, axis tight;
% caxis([-1e-6 0.5*max(RSS)]); colorbar
% xlabel(' m0, y-intercept');
% ylabel(' m1, slope');
% title(' Residual sum of squares');

% % PLOT GRADIENT HERE WITH quiver COMMAND

% \end{verbatim}
% \end{spacing}

%-------------------------------------------------------------
\end{document}
%-------------------------------------------------------------
